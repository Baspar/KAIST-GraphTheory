\chapter{Plane graphs}
    \section{General definition}
        \begin{description}
            \item[Plane graph] A plane graph is a graph $(V, E)$ such that:
                \begin{itemize}
                    \item $V\subseteq \mathbb{R}^2$ (vertices are part of the plane)
                    \item Every edge is an arc joining 2 vertices
                    \item The interior of each edge contains no vertex and no interior point of other edges
                \end{itemize}
            \item[Planar graph] A graph is planar if it is isomorphic to a plane graph.
            \item[Drawing] If $G$ is a surface, a drawing of a graph $G$ on $S$ is a mapping from $G$ so that vertices are mapped to distinct point of $S$, and each edge is mapped to an arc joining 2 points corresponding to the ends.\\
            \item[Map] A drawing without any edge crossing is called a map (or an embedding).\\
            \item[Embeddable] $G$ is ``embeddable'' on $S$ if there is \\
        \end{description}
    \section{Lemma}
        If $G$ is a 2-connected loopless plane graph, then every face boundary is a cycle of $G$, and each edge is incident with exactly 2 faces.
    \section{Euleur formula}
        \subsection{Euleur formula}
            If $G$ is a connected plane graph with :
            \begin{itemize}
                \item $v$ vertices
                \item $e$ edges
                \item $f$ faces
            \end{itemize}
            then $v-e+f=2$
        \subsection{$\searrow$Corrolary}
            If $G$ is a simple plane graph, then:\\
            $|E(G)|\leq 3|V(G)|-6$, unless $|V(G)|\leq 2$
        \subsection{$\searrow$Use example}
            \begin{itemize}
                \item $K_5$: $3-5-6< 10$ $\Rightarrow$ $K_5$ is nonplanar
            \end{itemize}
        \subsection{$\searrow$Corrolary}
            If $G$ is a simple plane graph with no triangle (cycle of length 3), then:\\
                $|E(G)|\leq 2|V(G)|-4$, unless $|V(G)|\leq 2$
    \section{Kuratowski theorem and $K_5$/$K_{3, 3}$ ``things''}
        \subsection{Kuratowski theorem}
            A graph is planar $\Leftrightarrow$ it has a $K_5$ or $K_{3, 3}$ topological minor.
        \subsection{$\searrow$Wagner-Kuratowski corrolary}
            A graph is planar $\Leftrightarrow$ it has a $K_5$ or $K_{3, 3}$ minor.
        \subsection{Convex embedding}
            \begin{itemize}
                \item Evey edge is mapped to a straight-line segment
                \item Every inner face is convex
                \item Then complement of the outer face is convex
            \end{itemize}
        \subsection{$\searrow$Property}
            If $G$ is a simple 3-connected graph with no topological minor isomorphic to $K_5$ or $K_{3, 3}$, then\\
            $G$ has a convex embedding on $\mathbb{R}^2$
        \subsection{Lemma}
            If $G/e$ has a $K_5$ topoligical minor, then $G$ has a $K_5$ OR $K_{3, 3}$ topoligical minor
        \subsection{Lemma}
        If $G/e$ has a $K_{3, 3}$ topoligical minor, then $G$ has a $K_{3, 3}$ topoligical minor
    \section{Dual graph}
        \subsection{Dual graph}
            For a given plane graph $G=(V, E)$, the geometric dual $G^*$ of $G$ is a graph $H$ such that:
            \begin{enumerate}
                \item every face of $G$ has precisely 1 vertex of $H$
                \item For every edge $e$ of $G$, $H$ has a unique edge $e^*$ that crosses $e$ and joins 2 vertices of $H$ that are in the faces incident with $e$, and no other edges of $H$ meet $e$
                \item $*:E(G)\rightarrow E(G^*)$ is a bijection
            \end{enumerate}
        \subsection{$\searrow$Lemma}
            If $G$ is a plane graph, then $G^*$ is connected.
        \subsection{$\searrow$Theorem}
            Let $G$ be a connected plane graph.\\
            If $H$ is the geometric dual of $G$, then $G$ is the geometric dual of $H$
        \subsection{Abstract dual}
            $H$ is called an abstract dual of $G$ if it exists a bijection
            \[*:E(G)\rightarrow E(H)\]
            such that 
            \[F\subseteq E(G)\text{ is the edge set of a cycle of }G\Leftrightarrow F^*\text{ is a minimal nonempty adjacent of }H\]
        \subsection{$\searrow$Property}
            If $H$ is the geometric dual of $G$, then $H$ is an abstract dual of $G$
        \subsection{$\searrow$Theorem}
            If $H$ is the abstract dual of $G$, then $H$ is an abstract dual of $G$
        \subsection{Whitney theorem}
            $G$ has an abstract dual $\Leftrightarrow$ $G$ is planar

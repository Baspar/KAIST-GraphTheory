\chapter{Nowhere-zero Flows}
    \section{Definition}
        \begin{description}
            \item[Circulation] Let $D$ be a directed graph.\\
                A circulation $\phi$ of $D$ is a function $\phi:E(D)\rightarrow \mathbb{R}$, such that for each vertex $v$ of $D$:
                \[
                    \sum_{e\in\delta^+(v)}\phi(e)=\sum_{e\in\delta^-(v)}\phi(e)
                \]
            \item[Nowhere zero $k$-flow] For a positive integer $k$, a nowhere zero $k$-flow of $D$ is a circulation $\phi$ such that:
                \[
                    |\phi(e)|\in \{1, 2, \ldots, k-1\}
                \]
            \item[Nowhere $\mathbb{Z}_k$-circulation] A $\mathbb{Z}_k$-circulation is a function $\phi:E(G)\rightarrow \mathbb{Z}_k$ such that:\\
                \[
                    \sum_{e\in\delta^+(e)}=\sum_{e\in\delta^-(e)}\text{ (mod k)}
                \]
            \item[Nowhere zero $\mathbb{Z}_k$-flow] A $\mathbb{Z}_k$-flow is a $\mathbb{Z}_k$-circulation such that $\phi(e)\neq0$ (mod k) for all edge $e$
            \item[Nowhere $\Gamma$-circulation] A $\mathbb{Z}_k$-circulation is a function $\phi:E(G)\rightarrow \Gamma$ such that:\\
                \[
                    \sum_{e\in\delta^+(e)}=\sum_{e\in\delta^-(e)}\text{ (mod k)}
                \]
            \item[Nowhere zero $\Gamma$-flow] A $\Gamma$-flow is a $\Gamma$-circulation such that $\phi(e)\neq0$ (mod k) for all edge $e$
        \end{description}
        (nb: $\mathbb{Z}_k = \{0, 1, \cdots, k-1\}$)
    \section{Theorem}
        Let $G$ be a plane graph. Then $G^*$(Geometric dual) is $k$-colorable $\Leftrightarrow$ $G$ has a nowhere zero $k$-flow\\
    \section{Theorem}
        $G$ has a nowhere zero $k$-flow $\Leftrightarrow$ $G$ ahs a nowhere zero $\mathbb{Z}_k$-flow
    \section{$\searrow$Corrolarry 1}
                if $k'>k$, and $g$ has a nowhere zero $k$-flow $\rightarrow$ $g$ has a nowhere zero $k'$-flow
    \section{$\searrow$Corrolarry 2 ($\mathbb{Z}_k$)}
        if $k'>k$, and $g$ has a nowhere zero $\mathbb{Z}_k$-flow $\rightarrow$ $g$ has a nowhere zero $\mathbb{Z}_{k'}$-flow
    \section{$\searrow$Corrolarry 3 ($\Gamma$)}
        $G$ has a nowhere zero $\Gamma$-flow $\Leftrightarrow$ $G$ has a 
    \section{$\searrow$Corrolarry 4 ($\Gamma$)}
        $|\Gamma_1|<|\Gamma_2|$. $G$ has a nowhere zero $\Gamma_1$-flow $\Rightarrow$ $G$ has a nowhere zero $\Gamma_2$-flow
    %TODO: complete missing CM
    \section{Seymour theorem}
        Every graph with no cut-edge has a nowhere zero 6-flow.
    \section{$\searrow$Proof}
        We may assume that:
        \begin{itemize}
            \item $G$ is connected $\Rightarrow$ $G$ is 2-edge-connected.\\
            \item $G$ is 3-edge-connected.(Cut edge size 2)
            \item $G$ is 3-regular.
        \end{itemize}
        $cl(X)$=mininum set $Y$ such that $Y\subseteq X$, and there is no cycle $C$with $|E(C)-Y|=1\text{ or }2$
        \begin{itemize}
            \item If $E(G)=cl(X)$ then there is a $\mathbb{Z}_3$-circulation $\phi$ such that $\phi(e)\neq 0 \forall e\notin X$
        \end{itemize}
        \textbf{Claim:} There is a set $X$ that is the edge set of the union of disjoints cycles such that $cl(X)=E(G)$
        Let $H$ be the largest connected subgraph of $G$ such that
        \[
            X\subseteq E(H)\subseteq cl(X)
        \]
        Choose $X$ such that $H$ is as big as possible.\\
        We will show that $H=G$.
        \begin{itemize}
            \item $H$ is an induced subgraph of G:\\
                $\Rightarrow$ $e\in cl(X)$
                $\Rightarrow$ $E(H+e)\subseteq cl(X)$
                $\Rightarrow$ Every vertex not in $V(H)$ has $\leq 1$ neighbor in $H$
            \item Otherwise: $H+e+f$ works
        \end{itemize}
        Blablabla$\ldots$


\chapter{Graph coloring}
    \section{Vertex coloring}
        \subsection{Definition}
            \begin{description}
                \item[k-coloring]: A $k$-coloring of a graph $G$ is a fonction $c:\nu(G)\rightarrow\{1, 2, \ldots, k\}$ such that $\forall$ edge $e=uv$, $c(u)\neq c(v)$
                \item[k-colorable]: A graph $G$ is k-colorable if it has a $k$-coloring.
                \item[$\chi$(G)]: The chromatic number $\chi(G)$ of a graph $G$ is the minimum $k$ such that $G$ has a $k$-coloring.
                \item[Stable set]: A set $X\subseteq V(G)$ is called stable if no 2 vertices in $X$ are adjacent
                \item[Maximum degree] Let's call $\Delta(G)=\max_{v\in V(G)}{deg_G(v)}$
            \end{description}
        \subsection{Loopfull G}
            If $G$ contains a loop, then $G$ is NOT colorable, and $\chi(G)=\infty$
        \subsection{Loopless G}
            If $G$ is loopless, then 
                \[\chi(G)\leq \Delta(G)+1\]
        \subsection{Theorem: Colorability of planar graph}
            Every simple planar graph is 5-colorable.
        \subsection{Theorem: Brook}
            If $G$ is simple, connected, then
                \[\chi(G)\leq\Delta(G)\]
            Unless $G$ is the complete graph or the odd cycle.
    \section{Edge coloring}
        \subsection{Definition}
            \begin{description}
                \item[k-edge-coloring] of a graph G is a function $c:E(G)\rightarrow \{1, 2, \cdots, k\}$ such that $c(e)\neq c(f)$ whenever e and f share a vertex.
                \item[k-edge-colorable] is a graph G if it has a k-edge-coloring
                \item[$\chi$'] is the edge chromatic number/index=min k such that G has a k-edge-coloring
            \end{description}
        \subsection{König's theorem}
            If $G$ is biparite, $\chi'(G)=\Delta(G)$
        \subsection{Vizing theorem}
            If $G$ is simple, then $\chi'(G)=\Delta(G)\text{ or }\Delta(G)+1$
        \subsection{$\searrow$Lemma}
            Let $G$ be a simple graph with $\Delta(G)\leq k$\\
            Let $v$ be a vertex of $G$\\
            Let $e_1, e_2, \cdots, e_n$ be some edges incident with $v$\\
            Then $G$ is k-edge-colorable
        \subsection{$\searrow$Prop}
            Lemma$\Rightarrow$Theorem: Take $k=\Delta(G)+1$
    \section{List coloring}
        \subsection{Definition}
            \begin{description}
                \item[List assignement] of a graph $G$ is a function that match each vertex $v$ to a list $L(v)$ of color
                \item[L-list coloring] of $G$ is a function on $V(G)$ such that$c(v)\in L(v)$ and $c(u)\neq c(v)$ whenever $u$ and $v$ are adjacent
                \item[k-list colorable] is a graph $G$ if for all list assigments $L$, with $|L(v)[\geq k \forall \text{ vertex }v$, $G$ has a L-list coloring
                \item[$\chi_l(G)$] = min k such that $G$ is k-list-colorable
            \end{description}
        \subsection{Theorem}
                $k-list-colorable \Rightarrow k-colorable$\\
                $k-list-colorable \not\Leftarrow k-colorable$\\
                $\chi(G)\leq\chi_l(G)$
        \subsection{Theorem}
            Every simple planar graph is $5-list-colorable$
        \subsection{Alan T~ Theorem}
            Every bipartite planar graph is $3-list-colorable$
        \subsection{List coloring conjecture}
            ``List edge-chromatic index'' $\Leftrightarrow$ $\chi_l'(G)$ or ch'$(G)$
        \subsection{Galvin's theorem}
            Let $G$ be a simple bipartite graph. Then:
            \[
                \Rightarrow \chi-l'(G)=\chi'(G)
            \]
    \section{Perfect graphs}
        \subsection{Definitions}
            \begin{description}
                \item[Clique] is a set of vertices pairwise adjacent
                \item[$\omega(G)$] is the maximum size of a clique in $G$. We know that $\chi(G)\geq\omega(G)$
            \end{description}
        \subsection{Theorem Erdois}
            For $k, l$, there is a graph $G$ such that $\chi(G)>k$ and $G$ has no cycle of length $\leq l$

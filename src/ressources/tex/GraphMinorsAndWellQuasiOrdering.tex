\chapter{Graph Minors and Well-Quasi-Ordering}
    \section{Quasi-order}
        A binary relation $(Q, \leq)$ is called a quasi-order if:
        \begin{itemize}
            \item reflexive ($a\leq a \forall a\in Q$)
            \item transitive ($a\leq b, b\leq c \Rightarrow a\leq c$)
        \end{itemize}
    \section{Wall-founded quasi-order}
        An quasi-order $(Q, \leq)$ is well-founded is $Q$ has no infinite strictly descending sequence:
        \[
            x_0>x_1>x_2>\cdots
        \]
        ($y<x \Leftrightarrow y\leq x\text{ and }x\not\leq y$)
        \subsection{Examples}
            \begin{itemize}
                \item $(\mathbb{Z}, \leq)$ is not well-founded
                \item $(\mathbb{N}, \leq)$ is well-founded
            \end{itemize}
    \section{Antichain}
        $X\subseteq Q$ is called an antichain if
        \[
            x\not\leq  y \forall x\neq y\in X$
        \]
    \section{Well-quasi-order}
        We say $(Q, \leq)$ is a well-quasi-order if:
        \begin{itemize}
            \item it is well-founded
            \item it has an infinite antichain
        \end{itemize}
        \subsection{Example}
            \begin{itemize}
                \item $(\mathbb{N}, \leq)$ is W.Q.O
                \item $(\mathbb{N}x\mathbb{N}, \leq)$ is W.Q.O
            \end{itemize}
    \section{Ramsey theorem (1930)}
        Let $k$ be a positive integer.\\
        For every $k$-edge-coloring of $K_\omega$ there exists a monochromatic infinite complete subgraph
        \subsection{Proof}
            Let $v_1, v_2, \cdots$  be the vertices of $K_\omega$\\
            Let $c(v_1)$=the color appearing at $v_1$ infinitely many times.\\
            By taking a subsequence, we may assume that all edge $vv_i(i>1)$ have color $c(v_1)$
            Let $c(v_2)$=the color appearing at $v_2$ infinitely many times in $v_2v_3, v_2v_4, \cdots$
            We may assume $v_2v_i(i>2)$ have the same color.
            Repeats\\
            We may assume the colour of $v_iv_j$ = the color of $v_iv_j=c(v_i)$
            There is a appearing infinitely many times in $c(v_1), c(v_2), c(v_3), \cdots$
    \section{Theorem}
        $(\mathbb{Q}, \leq)$ is a well-ordering-order\\
        $\Leftrightarrow$\\
        Every infinite sequence $x_1, x_2, \cdots$ has a pair $i, j$ such thath $i<j$ and $x_i<x_j$\\
        $\Leftrightarrow$\\
        Every infinite sequence $x_1, x_2, \cdots$ has an infinite subsequence $x_i_1\leq x_i_2\leq \cdots$ (with $i_1<i_2<\cdots$)
        \subsection{Proof}
        \begin{itemize}
            \item $3\Rightarrow 2\Rightarrow 1$ : trivial
            \item $1\Rightarrow 3$ : suppose $x_1, x_2, \cdots$ is an infinite sequence.\\
                We construct $K_\omega$ on $\{1, 2, \cdots\}$\\
                The color of $ij$ is:
                \begin{itemize}
                    \item 1 if $x_i\leq x_j$
                    \item 2 if $x_i> x_j$
                    \item 3 Otherwise
                \end{itemize}
                By infinite Ramsey, we conclude.\\
                Construct a W.Q.O. set, and let $(\mathbb{Q}_1, \leq_1)$ an $(\mathbb{Q}_2, \leq_2)$ be W.Q.O.
                \begin{enumerate}
                    \item The disjoint union $(Q_1\cap Q_2=\emptyset)$\\

                \end{enumerate}
        \end{itemize}
